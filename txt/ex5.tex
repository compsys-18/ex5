\documentclass{jarticle}
\usepackage{listings}
\lstset{
  basicstyle={\ttfamily},
  frame={tb}
}

\title{計算機システム演習 第5回レポート}
\author{17B13541 \and 細木隆豊}
\date{}

\begin{document}
\maketitle
  \section{実行結果}
    \begin{lstlisting}
    AND(0, 0) => 0
    AND(0, 1) => 0
    AND(1, 0) => 0
    AND(1, 1) => 1
    OR(0, 0) => 0
    OR(0, 1) => 1
    OR(1, 0) => 1
    OR(1, 1) => 1
    NOT(0) => 1
    NOT(1) => 0
    NAND(0, 0) => 1
    NAND(0, 1) => 1
    NAND(1, 0) => 1
    NAND(1, 1) => 0
    XOR(0, 0) => 0
    XOR(0, 1) => 1
    XOR(1, 0) => 1
    XOR(1, 1) => 0
    ANDN(11) => 1
    ANDN(10) => 0
    ANDN(01) => 0
    ANDN(00) => 0
    ORN(11) => 1
    ORN(10) => 1
    ORN(01) => 1
    ORN(00) => 0
    full_adder(in1 = 0, in2 = 0, carry_in = 0) =>
      (sum = 0, carry_out = 0)
    full_adder(in1 = 0, in2 = 1, carry_in = 0) =>
      (sum = 1, carry_out = 0)
    full_adder(in1 = 1, in2 = 0, carry_in = 0) =>
      (sum = 1, carry_out = 0)
    full_adder(in1 = 1, in2 = 1, carry_in = 0) =>
      (sum = 0, carry_out = 1)
    full_adder(in1 = 0, in2 = 0, carry_in = 1) =>
      (sum = 1, carry_out = 0)
    full_adder(in1 = 0, in2 = 1, carry_in = 1) =>
      (sum = 0, carry_out = 1)
    full_adder(in1 = 1, in2 = 0, carry_in = 1) =>
      (sum = 0, carry_out = 1)
    full_adder(in1 = 1, in2 = 1, carry_in = 1) =>
      (sum = 1, carry_out = 1)
    RCA(100, 200) => 300
    RCA(1073741824, 1073741823) => 2147483647
    RCA(1073741824, 1073741824)  => -2147483648
    RCA(-2147483648, -2147483648) => 0
    \end{lstlisting}
  \section{課題4の議論}
  {\large CLAの場合}

  PFAで$g_i$, $p_i$を計算する。\\
  $g_i = a_i \cdot b_i$,
  $p_i = a_i + b_i$
  で、これらは32bit全て並列で計算するので計1ゲート。

  4bitCLA内のCLUで$G_i$, $P_i$を計算する。\\
  $G_i = g_3 + p_3 \cdot g_2 + p_3 \cdot p_2 \cdot g_1 + p_3 \cdot p_2 \cdot p_1 \cdot g_0$,
  $P_i = p_0 \cdot p_1 \cdot p_2 \cdot p_3$
  よりこのときの最長パスは、$G_i$計算の4ゲートである。全8個の4bitCLAで並列計算できるので、計4ゲート。

  下位16bitCLAでCarryOutを計算する。\\
  $C_{out} = G_3 + G_2 \cdot P_3 + G_1 \cdot P_2 \cdot P_3 + G_0 \cdot P_1 \cdot P_2 \cdot P_3 + P_0 \cdot P_1 \cdot P_2 \cdot P_3 \cdot c_0$
  であるので、上での$G_i$, $P_i$計算を考慮するとこのパスは5ゲートである。

  上位16bitCLAで各4bitCLAへのCarryInを計算する。\\
  最長パスは、上と同じCarryOutの計算により5ゲートである。
  ただし最上位4bitCLAへのCarryIn計算には$G_i$計算回路と同じものを使うので、4ゲートである。
  他の4bitCLAへのCarryIn計算はこれ以下のゲートで求められる。

  4bitCLAで各桁へのCarryInを計算。\\
  最長パスは最上位bitへの繰り上がりで、
  $c_{3 + 4i} = g_{2 + 4i} + g_{1 + 4i} \cdot p_{2 + 4i} + g_{0 + 4i} \cdot p_{2 + 4i} \cdot p_{1 + 4i} + c_{0 + 4i} \cdot p_{2 + 4i} \cdot p_{1 + 4i} \cdot p_{0 + 4i}$
  を計算する4ゲートである。

  以上から合計はおよそ18ゲートであると考えられる。RCAと比較すると約1/5倍のゲートの通過数で実行できる。
  \section{感想}
  CLAでのゲート通過数の求め方がいまいちよくわからず、自分なりの解釈で課題4を解いてしまったので合っているかわかりませんが、CLAで行なっていることは理解できたと思っています。
\end{document}
